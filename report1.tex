\documentclass[dvipdfmx]{jsarticle}

\usepackage{cite}
\usepackage{amsmath,amssymb,amsfonts}
\usepackage{algorithmic}
\usepackage{graphicx}
\usepackage[dvipdfmx]{color}
\usepackage{textcomp}
\usepackage{xcolor}


\usepackage{algorithm}
\usepackage{ulem}
\usepackage{url}
\usepackage{outlines}
\usepackage{enumitem}
\usepackage{bm}
\usepackage{amsthm}
\usepackage{mathtools}
\usepackage{booktabs}
\usepackage{array}
\usepackage{amsbsy}
\errorcontextlines=10
\usepackage{float}


\renewcommand{\algorithmicrequire}{\textbf{Input:}}
\renewcommand{\algorithmicensure}{\textbf{Output:}}
\newcommand{\noPerp}{\mathop{\not\!\perp\!\!\!\perp} }
%\newcommand{\deffunc}{\mbox{1}\hspace{-0.25em}\mbox{l}}
\newcommand{\deffunc}{\mathbb{I}}


\let\hmmax\undefined %just to remove errors
\let\bmmax\undefined
\input{Definitions}

\newtheorem{thm}{Theorem}
\newtheorem{principle}{Principle}
\newcommand{\ctext}[1]{\raise0.2ex\hbox{\textcircled{\scriptsize{#1}}}}


% \def\BibTeX{{\rm B\kern-.05em{\sc i\kern-.025em b}\kern-.08em
%     T\kern-.1667em\lower.7ex\hbox{E}\kern-.125emX}}
    


\begin{document}

\section{提案手法}
サンプルサイズ$n$,$d$変数のデータセットの$z^n$

\begin{outline}
\1 構造方程式モデル(SEM)
    \2 $X = f(\text{Pa}(X)) + E_X (\bmod m_X)$
        \3 だたし,$E_X$は変数$X$の外生変数であり,変数$X$は$m_X$値の離散変数とする.
\1 スコア
    \2 尤度+関数の符号長
    \2 $- \log P(z^n; M, \hat{\bm{\theta}}(z^n)) + L(f)$
        \3 ただし,$L(f)$は関数$f: \text{Pa}(X) \to X$の符号長
        \3 このスコアはdecomposableなスコアである.つまり各ノードを子ノードとして考えることで,全ノードのスコアの和として書ける.
\1 DAG探索方法
    \2 GES(Greedy Equivalence Search)\cite{hauser2012characterization}に倣って,空のDAGからスタートしてエッジを付け加える・削除する・反転する操作を行いスコアを最大化させるDAGを求める.
\end{outline}

\section{実験結果}
\subsection{3変数}
\subsection{4変数}
\subsection{5変数}


\bibliography{main}
\bibliographystyle{tieice}

\end{document}
