\documentclass[dvipdfmx]{jsarticle}
\usepackage[dvipdfmx]{graphicx}
	\textheight 25.cm
	\textwidth 15cm
	\hoffset -2cm
	\voffset -2cm
	\oddsidemargin 3.0cm
	\evensidemargin 2.0cm
	\pagestyle{headings}
	
\usepackage{cite}
\usepackage{amsmath,amssymb,amsfonts}
\usepackage{algorithmic}
\usepackage{graphicx}
\usepackage[dvipdfmx]{color}
\usepackage{textcomp}
\usepackage{xcolor}


\usepackage{algorithm}
\usepackage{ulem}
\usepackage{url}
\usepackage{outlines}
\usepackage{enumitem}
\usepackage{bm}
\usepackage{amsthm}
\usepackage{mathtools}
\usepackage{booktabs}
\usepackage{array}
\usepackage{amsbsy}
\errorcontextlines=10
\usepackage{float}


\usepackage[pdf]{graphviz}

%%% Helper code for Overleaf's build system to
%%% automatically update output drawings when
%%% code in a \digraph{...} is modified
\usepackage{xpatch}
\makeatletter
\newcommand*{\addFileDependency}[1]{% argument=file name and extension
  \typeout{(#1)}
  \@addtofilelist{#1}
  \IfFileExists{#1}{}{\typeout{No file #1.}}
}
\makeatother
\xpretocmd{\digraph}{\addFileDependency{#2.dot}}{}{}


\renewcommand{\algorithmicrequire}{\textbf{Input:}}
\renewcommand{\algorithmicensure}{\textbf{Output:}}
\newcommand{\noPerp}{\mathop{\not\!\perp\!\!\!\perp} }
%\newcommand{\deffunc}{\mbox{1}\hspace{-0.25em}\mbox{l}}
\newcommand{\deffunc}{\mathbb{I}}


\let\hmmax\undefined %just to remove errors
\let\bmmax\undefined
\input{Definitions}

\newtheorem{thm}{Theorem}
\newtheorem{principle}{Principle}
\newcommand{\ctext}[1]{\raise0.2ex\hbox{\textcircled{\scriptsize{#1}}}}


% \def\BibTeX{{\rm B\kern-.05em{\sc i\kern-.025em b}\kern-.08em
%     T\kern-.1667em\lower.7ex\hbox{E}\kern-.125emX}}
    


\begin{document}

\section{提案手法}


\begin{outline}
尤度からなるスコアを最小化させるDAGを探索する,スコアベースのアルゴリズムである.未観測共通原因は考えない.
\1 構造方程式モデル(SEM)
    \2 $X_j = f_j(X_{\textbf{PA}_j}) + E_j (\bmod m_j)$
        \3 だたし,$X_{\textbf{PA}_j}$は変数$X_j$の親変数,$E_j$は変数$X_j$の外生変数であり,変数$X_j$及び外生変数$E_j$は$m_j$値の離散変数とし,$f_j$は関数$f_j: \mathcal{X}_{\textbf{Pa}_{j}} \to \mathcal{X}_j$である.
\1 スコア
    \2 尤度+関数の符号長
    \2 $- \log P(z^n; M, \hat{\bm{\theta}}(z^n)) + L(f)$
        \3 ただし,$L(f)$は関数$f_j: \{\mathcal{X}_{\textbf{Pa}_{j}}\} \to \mathcal{X}_j$の符号長$L(f_j)$を$j$について総和を取ったもの.
        \3 尤度も$j$に関する総和で書ける.
        \3 このスコアはdecomposableなスコアである.つまりグラフ全体のスコアは,各ノードとその親ノードをペアにして計算出来る局所的なスコアの総和として書ける.
\1 DAGの探索方法
    \2 GES(Greedy Equivalence Search)\cite{hauser2012characterization}に倣って,空のDAGからスタートしてエッジを付け加える・削除する・反転する操作を行いスコアを極小化させるDAGを求める.
\end{outline}

\section{実験と結果}
人工データを使って提案手法の推定精度を測定することで,提案手法がデータを生成した因果関係を同定する性能を実験的に検証した.サンプルサイズ$n$,$p$変数のデータセット$z^n$を100回生成して,正しく推論された因果グラフの割合を精度とした.$p$には$3,4,5$の場合を試した.また,以降の実験ではすべての離散変数$X$は10値の離散変数に設定した.
\subsection{3変数}
3変数$X_1, X_2, X_3$間のすべてのDAG(ループの無い因果関係)25通りのうち,対称性の観点からユニークなグラフ6通りを代表して選び,各DAGに対応して,加法誤差モデル$X_j = \sum_{k \in \textbf{PA}_j}f_{j, k}(X_k) + E_j (\bmod m_j)(j = 1, 2, 3)$により3変数$X_1, X_2, X_3$を生成させた.ここで$f_{j, k}$は関数$f: \mathcal{X}_k \to \mathcal{X}_j$であり,外生変数$E_j$は$K=10$値カテゴリカル分布$Cat(K=10, \bm{\pi})$から生成した.カテゴリカル分布のパラメータ$\bm{\pi}$は,データセット作成のたびに$\sum_i \pi_i = 1$の制約のもとでランダムに生成した.

表\ref{3vars_n=1000}に真の因果グラフとその推定精度の対応表を示す.推定精度には25通り全ての符号長を計算して最小符号長をもたらすDAGを選んだ「全探索」手法と,GESを用いたDAGの探索アルゴリズムを用いた「GES」の2つを載せている.直接の親変数を2つ持つような変数がある場合には推定精度が50\%前後となった.3変数の場合には,全探索した場合に比べてGESの精度は高々2\%の低下であった.

サンプルサイズを$n=10000$として,同様の実験を行った際の実験結果を表\ref{3vars_n=10000}に示す.一致性が確認できた.

\begin{table}[hbtp]
    \centering
    \caption{Results on 3 variables with $n=1000$.}
    \label{3vars_n=1000}
    \begin{tabular}[t]{|c||c|c|}
    \hline
    真の因果グラフ & GES & 全探索\\
    \hline \hline
    \digraph[scale=0.65]{dig01}{X1; X2; X3;} & 100 \% & 100 \% \\ \hline
    \digraph[scale=0.65]{dig02}{X1; X2->X3;} & 100 \% & 100 \% \\ \hline
    \digraph[scale=0.65]{dig03}{X1->X2; X1->X3;} & 100 \% & 100 \% \\ \hline
    \digraph[scale=0.65]{dig04}{X2->X1; X3->X1;} & 50 \% & 52 \% \\ \hline
    \digraph[scale=0.65]{dig05}{rankdir=LR; X1->X2; X1->X3; X2->X3;} & 45 \% & 47 \% \\ \hline
    \digraph[scale=0.65]{dig06}{rankdir=LR; X3->X1; X1->X2;} & 100 \% & 100 \% \\ \hline
    \end{tabular}
\end{table}


\begin{table}[hbtp]
    \centering
    \caption{Results on 3 variables with $n=10000$.}
    \label{3vars_n=10000}
    \begin{tabular}[t]{|c||c|c|}
    \hline
    真の因果グラフ & GES & 全探索\\
    \hline \hline
    \digraph[scale=0.65]{dig01}{X1; X2; X3;} & 100 \% & 100 \% \\ \hline
    \digraph[scale=0.65]{dig02}{X1; X2->X3;} & 100 \% & 100 \% \\ \hline
    \digraph[scale=0.65]{dig03}{X1->X2; X1->X3;} & 100 \% & 100 \% \\ \hline
    \digraph[scale=0.65]{dig04}{X2->X1; X3->X1;} & 100 \% & 100 \% \\ \hline
    \digraph[scale=0.65]{dig05}{rankdir=LR; X1->X2; X1->X3; X2->X3;} & 100 \% & 100 \% \\ \hline
    \digraph[scale=0.7]{dig06}{rankdir=LR; X3->X1; X1->X2;} & 100 \% & 100 \% \\ \hline
    \end{tabular}
\end{table}
\subsection{4変数}
複雑な因果関係を同定する性能を確かめるために,下のDAGの因果関係から4変数$X_1, X_2, X_3, X_4$を先の実験のSEMと同様に生成した.サンプルサイズは$n=1000$とした.推定精度は\textbf{48 \%}であった.

\digraph[scale=0.65]{dig07}{rankdir=LR; X1->X2; X1->X3; X3->X4; X3->X2}

\subsection{5変数}
原因を3つ持つ複雑な因果関係を同定する性能を確かめるために,下のDAGの因果関係から5変数$X_1, X_2, X_3, X_4, X_5$を3変数の実験のSEMと同様に生成した.$n=1000$のとき推定精度は\textbf{0 \%}であり,$n=10000$のとき推定精度は\textbf{57 \%}であった.

\digraph[scale=0.65]{dig08}{X1->X4; X2->X4; X3->X4;  X4->X5;}


\bibliography{main}
\bibliographystyle{tieice}

\end{document}
