\documentclass[dvipdfmx]{jsarticle}

\usepackage{cite}
\usepackage{amsmath,amssymb,amsfonts}
\usepackage{algorithmic}
\usepackage{graphicx}
\usepackage[dvipdfmx]{color}
\usepackage{textcomp}
\usepackage{xcolor}


\usepackage{algorithm}
\usepackage{ulem}
\usepackage{url}
\usepackage{outlines}
\usepackage{enumitem}
\usepackage{bm}
\usepackage{amsthm}
\usepackage{mathtools}
\usepackage{booktabs}
\usepackage{array}
\usepackage{amsbsy}
\errorcontextlines=10
\usepackage{float}


\usepackage[pdf]{graphviz}

%%% Helper code for Overleaf's build system to
%%% automatically update output drawings when
%%% code in a \digraph{...} is modified
\usepackage{xpatch}
\makeatletter
\newcommand*{\addFileDependency}[1]{% argument=file name and extension
  \typeout{(#1)}
  \@addtofilelist{#1}
  \IfFileExists{#1}{}{\typeout{No file #1.}}
}
\makeatother
\xpretocmd{\digraph}{\addFileDependency{#2.dot}}{}{}


\renewcommand{\algorithmicrequire}{\textbf{Input:}}
\renewcommand{\algorithmicensure}{\textbf{Output:}}
\newcommand{\noPerp}{\mathop{\not\!\perp\!\!\!\perp} }
%\newcommand{\deffunc}{\mbox{1}\hspace{-0.25em}\mbox{l}}
\newcommand{\deffunc}{\mathbb{I}}


\let\hmmax\undefined %just to remove errors
\let\bmmax\undefined
\input{Definitions}

\newtheorem{thm}{Theorem}
\newtheorem{principle}{Principle}
\newcommand{\ctext}[1]{\raise0.2ex\hbox{\textcircled{\scriptsize{#1}}}}


% \def\BibTeX{{\rm B\kern-.05em{\sc i\kern-.025em b}\kern-.08em
%     T\kern-.1667em\lower.7ex\hbox{E}\kern-.125emX}}
    


\begin{document}

\section{提案手法}


\begin{outline}
尤度からなるスコアを最大化させるDAGを探索する,スコアベースのアルゴリズムである.未観測共通原因は考えない.
\1 構造方程式モデル(SEM)
    \2 $X_j = f_j(X_{\textbf{PA}_j}) + E_j (\bmod m_j)$
        \3 だたし,$\textbf{PA}_j$は変数$X_j$の親変数,$E_j$は変数$X_j$の外生変数であり,変数$X_j$は$m_j$値の離散変数とする.
\1 スコア
    \2 尤度+関数の符号長
    \2 $- \log P(z^n; M, \hat{\bm{\theta}}(z^n)) + L(f)$
        \3 ただし,$L(f)$は関数$f: \text{Pa}(X) \to X$の符号長
        \3 このスコアはdecomposableなスコアである.つまり各ノードとその親ノードをペアにして計算出来る局所的なスコアに関して,グラフ全体のスコアを全ノードの局所的なスコアの総和として書ける.
\1 DAGの探索方法
    \2 GES(Greedy Equivalence Search)\cite{hauser2012characterization}に倣って,空のDAGからスタートしてエッジを付け加える・削除する・反転する操作を行いスコアを極大化させるDAGを求める.
\end{outline}

\section{実験と結果}
人工データを使って提案手法の推定精度を測定することで,提案手法がデータを生成した因果関係を同定する性能を実験的に検証した.サンプルサイズ$n$,$d$変数のデータセットの$z^n$を100回生成して,正しく推論された因果グラフの割合を精度とした.$d$には$3,4,5$の場合を試した.
\subsection{3変数}
3変数$X_1, X_2, X_3$
\begin{table}[hbtp]
    \centering
    \caption{Results on Abalone dataset.}
    \label{exp_abalone}
    \begin{tabular}[t]{|c||c|c|}
    \hline
    真の因果グラフ & GES & 全探索\\
    \hline \hline
    \digraph[scale=0.65]{dig01}{X1; X2; X3;} & 100 & 100 \\ \hline
    \digraph[scale=0.65]{dig02}{X1; X2->X3;} & 100 & 100 \\ \hline
    \digraph[scale=0.65]{dig03}{X1->X2; X1->X3;} & 100 & 100 \\ \hline
    \digraph[scale=0.65]{dig04}{X2->X1; X3->X1;} & 50 & 52 \\ \hline
    \digraph[scale=0.65]{dig05}{rankdir=LR; X1->X2; X1->X3; X2->X3;} & 45 & 47 \\ \hline
    \digraph[scale=0.65]{dig06}{rankdir=LR; X3->X1; X1->X2;} & 100 & 100 \\ \hline
    \end{tabular}
\end{table}
\subsection{4変数}
\subsection{5変数}


\bibliography{main}
\bibliographystyle{tieice}

\end{document}
